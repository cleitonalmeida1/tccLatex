%====================================================================================================
% FaultRecovery: A ampliação da biblioteca de tolerância a falhas
%====================================================================================================
% TCC
%----------------------------------------------------------------------------------------------------
% Autor				: Cleiton Gonçalves de Almeida
% Orientador		: Kleber Kruger
% Instituição 		: UFMS - Universidade Federal do Mato Grosso do Sul
% Departamento		: CPCX - Sistema de Informação
%----------------------------------------------------------------------------------------------------
% Data de criação	: 10 de Maio de 2016
%====================================================================================================

\chapter*{Resumo}

Atualmente, o ser humano utiliza diversos aparelhos eletrônicos, tais como celulares, tocadores de MP3, televisores, \textit{tablets}, e outros dispositivos usados no auxílio das atividades diárias e na melhoria da qualidade de vida. Graças a expansão da computação ubíqua, os sistemas embarcados que abrangem uma grande quantidade dos sistemas computacionais estão cada vez mais presentes no cotidiano das pessoas. No entanto esses sistemas podem apresentar defeitos, que indicam uma incapacidade do sistema executar uma determinada tarefa devido a erros em algum componente do dispositivo ou no ambiente, que por sua vez, são causados por falhas \cite{Nelson:1990}. Segundo Nelson \cite{Nelson:1990} uma falha é uma condição física anômala. As causas estão associadas a danos causados em algum componente, ferrugem, ou outros tipos de deteriorações; e perturbações externas, como duras condições ambientais, interferência eletromagnética, radiação ionizante, ou má utilização do sistema.

Os objetivos deste trabalho foram estudar possíveis causas de falhas em sistemas embarcados, modificar as bibliotecas \textit{FaultInjector} e \textit{FaultRecovery}. Inclusive criar uma classe de redundância de dados no qual sua função visa garantir a integridade dos dados de um sistema embarcado. Uma das modificações visa possibilitar ao usuário desenvolver uma máquina de estados, no qual cada estado pode ser implementado independentemente dos outros. Antes a \textit{FaultRecovery} não entregava ao usuário uma estrutura de desenvolvimento pronto, agora ela foi modificada para atender a um padrão de projeto chamado \textit{State}.

Ao final, são apresentados os resultados  mostrando o tempo de execução após as modificações realizadas na biblioteca \textit{FaultRecovery}, para verificar se essas alterações impactaram no desempenho do código testado. O teste realizado com a \textit{FaultRecovery} foi executado em 5,2107 segundos, enquanto que em média o teste sem a biblioteca foi executado em 4,6854 segundos, ou seja, a biblioteca elevou o tempo de execução do teste realizado em 0,5253 segundos. No entanto a biblioteca foi exposta a testes de recuperação de falhas, mostrando-se eficaz em todos eles. A classe também foi exposta a testes de desempenho e redundância de dados, nos resultados que não utilizaram a TData o tempo de execução médio foi de 0,0614 segundos, enquanto que nos testes com a TData o tempo médio foi de 0,3272 segundos, a classe TData elevou o tempo de execução do teste em 0,2658 segundos. No entanto no primeiro resultado a média de falhas encontradas foi de 44\% enquanto que no segundo foi de 0\%.

Como resultado deste trabalho, a ideia de modificação da biblioteca \textit{FaultRecovery} foi utilizada pelo projeto de extensão Coxim Robótica sediado na UFMS - Campus Coxim, no desenvolvimento de um programa para um carrinho seguidor de linha e continuará sendo utilizada em programas futuros. Também foi criada uma classe que possibilita a utilização de redundância de dados em um sistema embarcado, que foi inserida a biblioteca \textit{FaultRecovery}, que possibilita ao usuário definir se o seu sistema embarcado se recuperará de falhas e também poderá proteger seus dados mais importantes.